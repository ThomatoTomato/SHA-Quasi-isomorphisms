\documentclass[american,titlepage]{ntnuthesis}

% --- Macros --- %
    \newcommand{\startset}[1]{\{#1\}}
    \newcommand{\argument}{\hspace{0.05cm}\underline{\hspace{0.2cm}}\hspace{0.05cm}}
    \renewcommand{\tt}[1]{\text{#1}}

% --- Front page --- %
    \title{On the Derived Category of Strongly Homotopy Associative Algebras}
    \shorttitle{Derived SHA}
    \author{Thomas Wilskow Thorbjørnsen}
    \shortauthor{Thorbjørnsen}
    \date{\today}

% --- Bibliography --- %
    \addbibresource{thesis.bib}

% --- Glossary --- %
    \input{glossary.tex} % add glossary and acronym lists before document

% --- Extra bibs --- %
    \usepackage{xfrac}

% --- Folder structure --- %
    \usepackage{subfiles} 
    % Allows compilation of each subfile, rather to compile everything at once
    % Subfiles must have this file as option while using the documentclass subfile
    % Loading a subfile is done with \subfile{"path-to-file"}

% --- Styles --- %
    % \everymath{\displaystyle}

\begin{document}

    % --- Preface --- %
    \subfile{chapters/preface.tex}


    % --- Structure --- %
    \tableofcontents % This lists each listable section, subsection and subsubsection 

    \printglossary[type=\acronymtype] % Print acronyms
    \printglossary                    % Print glossary


    % --- Chapter 1 --- %
    \chapter{Bar and Cobar Construction}
        \subfile{chapters/twisting.tex}

    % --- Chapter 2 --- %
    \chapter{Homotopy Theory of Algebras}
        \subfile{chapters/models.tex}

    % --- Chapter 3 --- %
    \chapter{Derived Categories of Strongly Homotopy Associative Algebras}
        \subfile{chapters/derived.tex}

    % --- Bibliography --- %
    \chapter*{\bibname}
    \printbibliography[heading=none]


    \appendix
    % --- Appendix A --- %
    \chapter{Simplicial Objects}
        \subfile{appendices/simplex.tex}
    
    % --- Appendix B --- %
    \chapter{Spectral Sequences}
        \subfile{appendices/spectral.tex}

    % --- Appendix C --- %
    \chapter{Symmetric Monoidal Categories}
        \subfile{appendices/symmetric.tex}

\end{document}
