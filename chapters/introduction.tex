\documentclass[../thesis.tex]{subfiles}

\begin{document}

    \newpage
    
    \chapter*{Introduction}

        A differential graded algebra, or simply dg-algebra, is an associative algebra where the underlying object is a cochain complex. Any dg-algebra $A$ naturally carries homotopical information, and we get a graded algebra by considering the homology algebra $H^*A$. When we are working with homology algebras, there are many more morphisms than the morphisms coming from the differential graded structure. To understand homology algebras in the context of their dg-counterparts, we should restrict our attention solely to those morphisms from this structure. This leads us to the definition of a quasi-isomorphism, that is, morphisms $f:A\rightarrow B$ between dg-algebras such that $H^*f:H^*A\rightarrow H^*B$ is an isomorphism.

        Localization is involved when constructing this category of homology algebras $\tt{HoAlg}_\mathbb{K}^\bullet$. We say that
        \begin{align*}
            \tt{HoAlg}_\mathbb{K}^\bullet = \tt{Alg}_\mathbb{K}^\bullet[\tt{Qis}^{-1}]\tt{.}
        \end{align*}
        Localization works by adding morphisms, and we add new morphisms such that at least the intended class of morphisms we want to be invertible is invertible. The problem with this is that controlling how many morphisms we add is difficult, so figuring out which dg-algebras are quasi-isomorphic is not a simple process.

        There is a weaker structure called strongly homotopy associative algebras, or $A_\infty$-algebras. An $A_\infty$-algebra is almost a dg-algebra, but the multiplication may fail to be associative. Instead, we assume that the associator is null-homotopic and an infinite hierarchy of homotopies controls this homotopy. By considering an $A_\infty$-algebra $A$ up to homotopy, we see that the homotopy algebra $A$ defines a graded algebra.

        It is becoming well known that quasi-isomorphisms $f:A\rightarrow B$ between $A_\infty$-algebras admit a homotopy inverse. When we localize the category of $A_\infty$-algebras at quasi-isomorphism, there is an equivalence to the homotopy category
        \begin{align*}
            \tt{HoAlg}_\infty = \tt{Alg}_\infty[\tt{Qis}^{-1}] \simeq \sfrac{\tt{Alg}_\infty}{\sim}\tt{.}
        \end{align*}
        Using this construction, we can bypass the localization construction. Instead of adding new morphisms to invert the quasi-isomorphisms, we can identify homotopic morphisms.

        What might be surprising is that there is an equivalence of categories,
        \begin{align*}
            \tt{HoAlg}_\mathbb{K}^\bullet \simeq \tt{HoAlg}_\infty\tt{.}
        \end{align*}
        This equivalence is given by localizing the non-full inclusion functor $i: \tt{Alg}_\mathbb{K}^\bullet \rightarrow \tt{Alg}_\infty$ at quasi-isomorphisms. We may say that a quasi-isomorphism $f: A\rightarrow B$ between dg-algebras admits a homotopy inverse of the corresponding $A_\infty$-algebras. Similarly, we bypass the localization construction by considering homotopy algebras,
        \begin{align*}
            \tt{HoAlg}_\mathbb{K} \simeq \sfrac{\tt{Alg}_\infty}{\sim}\tt{,}
        \end{align*}

        This result is still true if we consider quasi-isomorphisms $f:M\rightarrow N$ between $A$-modules. If we consider $M$ and $N$ as $A$-polydules, that is, $A_\infty$-modules, the morphism $f$ admits a homotopy inverse. With this in mind, there are equivalences of categories,
        \begin{align*}
            D_\infty A \simeq K_\infty A \simeq DA\tt{.}
        \end{align*}
        Here, $D_\infty A$ and $K_\infty A$ denote the derived and homotopy category of the category of $A_\infty$-modules, respectively.

        In this thesis, we investigate a proof provided by Lef\`evre-Hasegawa \cite{LefevreHasegawa03} on the homotopy invertibility of quasi-isomorphisms. In our approach, we will take a lot of inspiration from Loday and Vallette \cite{Loday12}. We wish to elaborate upon Lef\`evre-Hasegawa's work to make this particular instance clearer and more accessible. Many of the concepts we will discuss here for associative algebras have been generalized to many different algebras. See, for instance, \cite{Vallette20} for a generalization to Koszul operads.
        
        The thesis is split into three different chapters.

        \subsubsection*{Chapter 1 - The Bar and Cobar Construction}
            In Chapter 1, we develop the theory of dg-algebras and dg-coalgebras. We try to make the theory of coalgebras more intuitive by comparing how they differ from algebras. The augmented algebras and conilpotent coalgebras are of utmost importance in this thesis.

            The essential tool developed in this chapter is the bar and cobar construction, denoted as $B$ and $\Omega$, respectively. Twisting morphisms play a unique role as they define a functor, represented by the bar and cobar construction. Thus, we have an adjoint pair of functors,
            \begin{center}
                \begin{tikzcd}
                    \tt{coAlg}_{\mathbb{K}, \tt{conil}}^\bullet \ar[yshift = 1ex]{r}[]{\Omega} \ar[phantom]{r}[]{\bot} & \tt{Alg}_{\mathbb{K},+}^\bullet \ar[yshift = -1ex]{l}[below]{B}
                \end{tikzcd}
            \end{center}

            Lastly, we define $A_\infty$-algebras in terms of the bar construction. We will think of these as the algebras which make the bar construction fully faithful on the image of quasi-free conilpotent dg-coalgebras. We can thus think of an $A_\infty$-algebra in two different ways, either as a dg-algebra with strong homotopy associativity or as a conilpotent dg-coalgebra. Both points of view will be fruitful.

        \subsubsection*{Chapter 2 - Homotopy Theory of Algebras}
            Chapter 2 aims to explain some of the homotopy theories of dg-algebras, conilpotent dg-coalgebras, and $A_\infty$-algebras. We start by giving an exposition on model categories, having a special interest in Whitehead's theorem, the fundamental theorem of model categories, and Quillen equivalences.

            We upgrade the cobar-bar adjunction into a Quillen equivalence, identifying the homotopy category of dg-algebras and conilpotent dg-coalgebras. The category of $A_\infty$-algebra will be equivalent to the bifibrant conilpotent dg-coalgebras. This will allow us to show the first claim,
            \begin{align*}
                \tt{HoAlg}_\mathbb{K} \simeq \sfrac{\tt{Alg}_\infty}{\sim}\tt{.}
            \end{align*}

        \subsubsection*{Chapter 3 - Derived Categories of Strongly Homotopy Associative Algebras}
            In the final chapter, we investigate the homotopy theory of modules over dg-algebras and comodules over dg-coalgebras. We will further develop the theory of twisting morphisms to obtain Quillen equivalences,

            \begin{center}
                \begin{tikzcd}
                    \tt{coMod}^C \ar[yshift = 1ex]{r}[]{L_\alpha} \ar[phantom]{r}[]{\bot} & \tt{Mod}^A \ar[yshift = -1ex]{l}[below]{R_\alpha}
                \end{tikzcd}
            \end{center}

            We prove the fundamental theorem of twisting morphisms, which allows us to characterize whenever a twisting morphism defines a Quillen equivalence.

            $A_\infty$-modules of $A$, called $A$-polydules are defined to be objects being the converse of $R_\alpha$ whenever $C = BA$. We may then see that $A$-polydules are the bifibrant $BA$-comodules. We will then define the derived category of polydules, $D_\infty A$. We will conclude the thesis by showing that,
            \begin{align*}
                D_\infty A \simeq K_\infty A \simeq DA\tt{.}
            \end{align*}

        \subsubsection*{Prerequisites}
            We assume the reader is familiar with homological algebra, category theory, triangulated categories, and Kan extensions. The theory of monads, simplicial sets, spectral sequences, and symmetric monoidal categories will also be applied. At the end of the thesis, four appendixes are supplied, recalling the definitions and most important results, which we will use throughout this thesis.

\end{document}